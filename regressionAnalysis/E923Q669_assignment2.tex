\documentclass[11pt]{article}

    \usepackage[breakable]{tcolorbox}
    \usepackage{parskip} % Stop auto-indenting (to mimic markdown behaviour)
    

    % Basic figure setup, for now with no caption control since it's done
    % automatically by Pandoc (which extracts ![](path) syntax from Markdown).
    \usepackage{graphicx}
    % Maintain compatibility with old templates. Remove in nbconvert 6.0
    \let\Oldincludegraphics\includegraphics
    % Ensure that by default, figures have no caption (until we provide a
    % proper Figure object with a Caption API and a way to capture that
    % in the conversion process - todo).
    \usepackage{caption}
    \DeclareCaptionFormat{nocaption}{}
    \captionsetup{format=nocaption,aboveskip=0pt,belowskip=0pt}

    \usepackage{float}
    \floatplacement{figure}{H} % forces figures to be placed at the correct location
    \usepackage{xcolor} % Allow colors to be defined
    \usepackage{enumerate} % Needed for markdown enumerations to work
    \usepackage{geometry} % Used to adjust the document margins
    \usepackage{amsmath} % Equations
    \usepackage{amssymb} % Equations
    \usepackage{textcomp} % defines textquotesingle
    % Hack from http://tex.stackexchange.com/a/47451/13684:
    \AtBeginDocument{%
        \def\PYZsq{\textquotesingle}% Upright quotes in Pygmentized code
    }
    \usepackage{upquote} % Upright quotes for verbatim code
    \usepackage{eurosym} % defines \euro

    \usepackage{iftex}
    \ifPDFTeX
        \usepackage[T1]{fontenc}
        \IfFileExists{alphabeta.sty}{
              \usepackage{alphabeta}
          }{
              \usepackage[mathletters]{ucs}
              \usepackage[utf8x]{inputenc}
          }
    \else
        \usepackage{fontspec}
        \usepackage{unicode-math}
    \fi

    \usepackage{fancyvrb} % verbatim replacement that allows latex
    \usepackage{grffile} % extends the file name processing of package graphics
                         % to support a larger range
    \makeatletter % fix for old versions of grffile with XeLaTeX
    \@ifpackagelater{grffile}{2019/11/01}
    {
      % Do nothing on new versions
    }
    {
      \def\Gread@@xetex#1{%
        \IfFileExists{"\Gin@base".bb}%
        {\Gread@eps{\Gin@base.bb}}%
        {\Gread@@xetex@aux#1}%
      }
    }
    \makeatother
    \usepackage[Export]{adjustbox} % Used to constrain images to a maximum size
    \adjustboxset{max size={0.9\linewidth}{0.9\paperheight}}

    % The hyperref package gives us a pdf with properly built
    % internal navigation ('pdf bookmarks' for the table of contents,
    % internal cross-reference links, web links for URLs, etc.)
    \usepackage{hyperref}
    % The default LaTeX title has an obnoxious amount of whitespace. By default,
    % titling removes some of it. It also provides customization options.
    \usepackage{titling}
    \usepackage{longtable} % longtable support required by pandoc >1.10
    \usepackage{booktabs}  % table support for pandoc > 1.12.2
    \usepackage{array}     % table support for pandoc >= 2.11.3
    \usepackage{calc}      % table minipage width calculation for pandoc >= 2.11.1
    \usepackage[inline]{enumitem} % IRkernel/repr support (it uses the enumerate* environment)
    \usepackage[normalem]{ulem} % ulem is needed to support strikethroughs (\sout)
                                % normalem makes italics be italics, not underlines
    \usepackage{soul}      % strikethrough (\st) support for pandoc >= 3.0.0
    \usepackage{mathrsfs}
    

    
    % Colors for the hyperref package
    \definecolor{urlcolor}{rgb}{0,.145,.698}
    \definecolor{linkcolor}{rgb}{.71,0.21,0.01}
    \definecolor{citecolor}{rgb}{.12,.54,.11}

    % ANSI colors
    \definecolor{ansi-black}{HTML}{3E424D}
    \definecolor{ansi-black-intense}{HTML}{282C36}
    \definecolor{ansi-red}{HTML}{E75C58}
    \definecolor{ansi-red-intense}{HTML}{B22B31}
    \definecolor{ansi-green}{HTML}{00A250}
    \definecolor{ansi-green-intense}{HTML}{007427}
    \definecolor{ansi-yellow}{HTML}{DDB62B}
    \definecolor{ansi-yellow-intense}{HTML}{B27D12}
    \definecolor{ansi-blue}{HTML}{208FFB}
    \definecolor{ansi-blue-intense}{HTML}{0065CA}
    \definecolor{ansi-magenta}{HTML}{D160C4}
    \definecolor{ansi-magenta-intense}{HTML}{A03196}
    \definecolor{ansi-cyan}{HTML}{60C6C8}
    \definecolor{ansi-cyan-intense}{HTML}{258F8F}
    \definecolor{ansi-white}{HTML}{C5C1B4}
    \definecolor{ansi-white-intense}{HTML}{A1A6B2}
    \definecolor{ansi-default-inverse-fg}{HTML}{FFFFFF}
    \definecolor{ansi-default-inverse-bg}{HTML}{000000}

    % common color for the border for error outputs.
    \definecolor{outerrorbackground}{HTML}{FFDFDF}

    % commands and environments needed by pandoc snippets
    % extracted from the output of `pandoc -s`
    \providecommand{\tightlist}{%
      \setlength{\itemsep}{0pt}\setlength{\parskip}{0pt}}
    \DefineVerbatimEnvironment{Highlighting}{Verbatim}{commandchars=\\\{\}}
    % Add ',fontsize=\small' for more characters per line
    \newenvironment{Shaded}{}{}
    \newcommand{\KeywordTok}[1]{\textcolor[rgb]{0.00,0.44,0.13}{\textbf{{#1}}}}
    \newcommand{\DataTypeTok}[1]{\textcolor[rgb]{0.56,0.13,0.00}{{#1}}}
    \newcommand{\DecValTok}[1]{\textcolor[rgb]{0.25,0.63,0.44}{{#1}}}
    \newcommand{\BaseNTok}[1]{\textcolor[rgb]{0.25,0.63,0.44}{{#1}}}
    \newcommand{\FloatTok}[1]{\textcolor[rgb]{0.25,0.63,0.44}{{#1}}}
    \newcommand{\CharTok}[1]{\textcolor[rgb]{0.25,0.44,0.63}{{#1}}}
    \newcommand{\StringTok}[1]{\textcolor[rgb]{0.25,0.44,0.63}{{#1}}}
    \newcommand{\CommentTok}[1]{\textcolor[rgb]{0.38,0.63,0.69}{\textit{{#1}}}}
    \newcommand{\OtherTok}[1]{\textcolor[rgb]{0.00,0.44,0.13}{{#1}}}
    \newcommand{\AlertTok}[1]{\textcolor[rgb]{1.00,0.00,0.00}{\textbf{{#1}}}}
    \newcommand{\FunctionTok}[1]{\textcolor[rgb]{0.02,0.16,0.49}{{#1}}}
    \newcommand{\RegionMarkerTok}[1]{{#1}}
    \newcommand{\ErrorTok}[1]{\textcolor[rgb]{1.00,0.00,0.00}{\textbf{{#1}}}}
    \newcommand{\NormalTok}[1]{{#1}}

    % Additional commands for more recent versions of Pandoc
    \newcommand{\ConstantTok}[1]{\textcolor[rgb]{0.53,0.00,0.00}{{#1}}}
    \newcommand{\SpecialCharTok}[1]{\textcolor[rgb]{0.25,0.44,0.63}{{#1}}}
    \newcommand{\VerbatimStringTok}[1]{\textcolor[rgb]{0.25,0.44,0.63}{{#1}}}
    \newcommand{\SpecialStringTok}[1]{\textcolor[rgb]{0.73,0.40,0.53}{{#1}}}
    \newcommand{\ImportTok}[1]{{#1}}
    \newcommand{\DocumentationTok}[1]{\textcolor[rgb]{0.73,0.13,0.13}{\textit{{#1}}}}
    \newcommand{\AnnotationTok}[1]{\textcolor[rgb]{0.38,0.63,0.69}{\textbf{\textit{{#1}}}}}
    \newcommand{\CommentVarTok}[1]{\textcolor[rgb]{0.38,0.63,0.69}{\textbf{\textit{{#1}}}}}
    \newcommand{\VariableTok}[1]{\textcolor[rgb]{0.10,0.09,0.49}{{#1}}}
    \newcommand{\ControlFlowTok}[1]{\textcolor[rgb]{0.00,0.44,0.13}{\textbf{{#1}}}}
    \newcommand{\OperatorTok}[1]{\textcolor[rgb]{0.40,0.40,0.40}{{#1}}}
    \newcommand{\BuiltInTok}[1]{{#1}}
    \newcommand{\ExtensionTok}[1]{{#1}}
    \newcommand{\PreprocessorTok}[1]{\textcolor[rgb]{0.74,0.48,0.00}{{#1}}}
    \newcommand{\AttributeTok}[1]{\textcolor[rgb]{0.49,0.56,0.16}{{#1}}}
    \newcommand{\InformationTok}[1]{\textcolor[rgb]{0.38,0.63,0.69}{\textbf{\textit{{#1}}}}}
    \newcommand{\WarningTok}[1]{\textcolor[rgb]{0.38,0.63,0.69}{\textbf{\textit{{#1}}}}}


    % Define a nice break command that doesn't care if a line doesn't already
    % exist.
    \def\br{\hspace*{\fill} \\* }
    % Math Jax compatibility definitions
    \def\gt{>}
    \def\lt{<}
    \let\Oldtex\TeX
    \let\Oldlatex\LaTeX
    \renewcommand{\TeX}{\textrm{\Oldtex}}
    \renewcommand{\LaTeX}{\textrm{\Oldlatex}}
    % Document parameters
    % Document title
    \title{E923Q669\_assignment2}
    
    
    
    
    
    
    
% Pygments definitions
\makeatletter
\def\PY@reset{\let\PY@it=\relax \let\PY@bf=\relax%
    \let\PY@ul=\relax \let\PY@tc=\relax%
    \let\PY@bc=\relax \let\PY@ff=\relax}
\def\PY@tok#1{\csname PY@tok@#1\endcsname}
\def\PY@toks#1+{\ifx\relax#1\empty\else%
    \PY@tok{#1}\expandafter\PY@toks\fi}
\def\PY@do#1{\PY@bc{\PY@tc{\PY@ul{%
    \PY@it{\PY@bf{\PY@ff{#1}}}}}}}
\def\PY#1#2{\PY@reset\PY@toks#1+\relax+\PY@do{#2}}

\@namedef{PY@tok@w}{\def\PY@tc##1{\textcolor[rgb]{0.73,0.73,0.73}{##1}}}
\@namedef{PY@tok@c}{\let\PY@it=\textit\def\PY@tc##1{\textcolor[rgb]{0.24,0.48,0.48}{##1}}}
\@namedef{PY@tok@cp}{\def\PY@tc##1{\textcolor[rgb]{0.61,0.40,0.00}{##1}}}
\@namedef{PY@tok@k}{\let\PY@bf=\textbf\def\PY@tc##1{\textcolor[rgb]{0.00,0.50,0.00}{##1}}}
\@namedef{PY@tok@kp}{\def\PY@tc##1{\textcolor[rgb]{0.00,0.50,0.00}{##1}}}
\@namedef{PY@tok@kt}{\def\PY@tc##1{\textcolor[rgb]{0.69,0.00,0.25}{##1}}}
\@namedef{PY@tok@o}{\def\PY@tc##1{\textcolor[rgb]{0.40,0.40,0.40}{##1}}}
\@namedef{PY@tok@ow}{\let\PY@bf=\textbf\def\PY@tc##1{\textcolor[rgb]{0.67,0.13,1.00}{##1}}}
\@namedef{PY@tok@nb}{\def\PY@tc##1{\textcolor[rgb]{0.00,0.50,0.00}{##1}}}
\@namedef{PY@tok@nf}{\def\PY@tc##1{\textcolor[rgb]{0.00,0.00,1.00}{##1}}}
\@namedef{PY@tok@nc}{\let\PY@bf=\textbf\def\PY@tc##1{\textcolor[rgb]{0.00,0.00,1.00}{##1}}}
\@namedef{PY@tok@nn}{\let\PY@bf=\textbf\def\PY@tc##1{\textcolor[rgb]{0.00,0.00,1.00}{##1}}}
\@namedef{PY@tok@ne}{\let\PY@bf=\textbf\def\PY@tc##1{\textcolor[rgb]{0.80,0.25,0.22}{##1}}}
\@namedef{PY@tok@nv}{\def\PY@tc##1{\textcolor[rgb]{0.10,0.09,0.49}{##1}}}
\@namedef{PY@tok@no}{\def\PY@tc##1{\textcolor[rgb]{0.53,0.00,0.00}{##1}}}
\@namedef{PY@tok@nl}{\def\PY@tc##1{\textcolor[rgb]{0.46,0.46,0.00}{##1}}}
\@namedef{PY@tok@ni}{\let\PY@bf=\textbf\def\PY@tc##1{\textcolor[rgb]{0.44,0.44,0.44}{##1}}}
\@namedef{PY@tok@na}{\def\PY@tc##1{\textcolor[rgb]{0.41,0.47,0.13}{##1}}}
\@namedef{PY@tok@nt}{\let\PY@bf=\textbf\def\PY@tc##1{\textcolor[rgb]{0.00,0.50,0.00}{##1}}}
\@namedef{PY@tok@nd}{\def\PY@tc##1{\textcolor[rgb]{0.67,0.13,1.00}{##1}}}
\@namedef{PY@tok@s}{\def\PY@tc##1{\textcolor[rgb]{0.73,0.13,0.13}{##1}}}
\@namedef{PY@tok@sd}{\let\PY@it=\textit\def\PY@tc##1{\textcolor[rgb]{0.73,0.13,0.13}{##1}}}
\@namedef{PY@tok@si}{\let\PY@bf=\textbf\def\PY@tc##1{\textcolor[rgb]{0.64,0.35,0.47}{##1}}}
\@namedef{PY@tok@se}{\let\PY@bf=\textbf\def\PY@tc##1{\textcolor[rgb]{0.67,0.36,0.12}{##1}}}
\@namedef{PY@tok@sr}{\def\PY@tc##1{\textcolor[rgb]{0.64,0.35,0.47}{##1}}}
\@namedef{PY@tok@ss}{\def\PY@tc##1{\textcolor[rgb]{0.10,0.09,0.49}{##1}}}
\@namedef{PY@tok@sx}{\def\PY@tc##1{\textcolor[rgb]{0.00,0.50,0.00}{##1}}}
\@namedef{PY@tok@m}{\def\PY@tc##1{\textcolor[rgb]{0.40,0.40,0.40}{##1}}}
\@namedef{PY@tok@gh}{\let\PY@bf=\textbf\def\PY@tc##1{\textcolor[rgb]{0.00,0.00,0.50}{##1}}}
\@namedef{PY@tok@gu}{\let\PY@bf=\textbf\def\PY@tc##1{\textcolor[rgb]{0.50,0.00,0.50}{##1}}}
\@namedef{PY@tok@gd}{\def\PY@tc##1{\textcolor[rgb]{0.63,0.00,0.00}{##1}}}
\@namedef{PY@tok@gi}{\def\PY@tc##1{\textcolor[rgb]{0.00,0.52,0.00}{##1}}}
\@namedef{PY@tok@gr}{\def\PY@tc##1{\textcolor[rgb]{0.89,0.00,0.00}{##1}}}
\@namedef{PY@tok@ge}{\let\PY@it=\textit}
\@namedef{PY@tok@gs}{\let\PY@bf=\textbf}
\@namedef{PY@tok@gp}{\let\PY@bf=\textbf\def\PY@tc##1{\textcolor[rgb]{0.00,0.00,0.50}{##1}}}
\@namedef{PY@tok@go}{\def\PY@tc##1{\textcolor[rgb]{0.44,0.44,0.44}{##1}}}
\@namedef{PY@tok@gt}{\def\PY@tc##1{\textcolor[rgb]{0.00,0.27,0.87}{##1}}}
\@namedef{PY@tok@err}{\def\PY@bc##1{{\setlength{\fboxsep}{\string -\fboxrule}\fcolorbox[rgb]{1.00,0.00,0.00}{1,1,1}{\strut ##1}}}}
\@namedef{PY@tok@kc}{\let\PY@bf=\textbf\def\PY@tc##1{\textcolor[rgb]{0.00,0.50,0.00}{##1}}}
\@namedef{PY@tok@kd}{\let\PY@bf=\textbf\def\PY@tc##1{\textcolor[rgb]{0.00,0.50,0.00}{##1}}}
\@namedef{PY@tok@kn}{\let\PY@bf=\textbf\def\PY@tc##1{\textcolor[rgb]{0.00,0.50,0.00}{##1}}}
\@namedef{PY@tok@kr}{\let\PY@bf=\textbf\def\PY@tc##1{\textcolor[rgb]{0.00,0.50,0.00}{##1}}}
\@namedef{PY@tok@bp}{\def\PY@tc##1{\textcolor[rgb]{0.00,0.50,0.00}{##1}}}
\@namedef{PY@tok@fm}{\def\PY@tc##1{\textcolor[rgb]{0.00,0.00,1.00}{##1}}}
\@namedef{PY@tok@vc}{\def\PY@tc##1{\textcolor[rgb]{0.10,0.09,0.49}{##1}}}
\@namedef{PY@tok@vg}{\def\PY@tc##1{\textcolor[rgb]{0.10,0.09,0.49}{##1}}}
\@namedef{PY@tok@vi}{\def\PY@tc##1{\textcolor[rgb]{0.10,0.09,0.49}{##1}}}
\@namedef{PY@tok@vm}{\def\PY@tc##1{\textcolor[rgb]{0.10,0.09,0.49}{##1}}}
\@namedef{PY@tok@sa}{\def\PY@tc##1{\textcolor[rgb]{0.73,0.13,0.13}{##1}}}
\@namedef{PY@tok@sb}{\def\PY@tc##1{\textcolor[rgb]{0.73,0.13,0.13}{##1}}}
\@namedef{PY@tok@sc}{\def\PY@tc##1{\textcolor[rgb]{0.73,0.13,0.13}{##1}}}
\@namedef{PY@tok@dl}{\def\PY@tc##1{\textcolor[rgb]{0.73,0.13,0.13}{##1}}}
\@namedef{PY@tok@s2}{\def\PY@tc##1{\textcolor[rgb]{0.73,0.13,0.13}{##1}}}
\@namedef{PY@tok@sh}{\def\PY@tc##1{\textcolor[rgb]{0.73,0.13,0.13}{##1}}}
\@namedef{PY@tok@s1}{\def\PY@tc##1{\textcolor[rgb]{0.73,0.13,0.13}{##1}}}
\@namedef{PY@tok@mb}{\def\PY@tc##1{\textcolor[rgb]{0.40,0.40,0.40}{##1}}}
\@namedef{PY@tok@mf}{\def\PY@tc##1{\textcolor[rgb]{0.40,0.40,0.40}{##1}}}
\@namedef{PY@tok@mh}{\def\PY@tc##1{\textcolor[rgb]{0.40,0.40,0.40}{##1}}}
\@namedef{PY@tok@mi}{\def\PY@tc##1{\textcolor[rgb]{0.40,0.40,0.40}{##1}}}
\@namedef{PY@tok@il}{\def\PY@tc##1{\textcolor[rgb]{0.40,0.40,0.40}{##1}}}
\@namedef{PY@tok@mo}{\def\PY@tc##1{\textcolor[rgb]{0.40,0.40,0.40}{##1}}}
\@namedef{PY@tok@ch}{\let\PY@it=\textit\def\PY@tc##1{\textcolor[rgb]{0.24,0.48,0.48}{##1}}}
\@namedef{PY@tok@cm}{\let\PY@it=\textit\def\PY@tc##1{\textcolor[rgb]{0.24,0.48,0.48}{##1}}}
\@namedef{PY@tok@cpf}{\let\PY@it=\textit\def\PY@tc##1{\textcolor[rgb]{0.24,0.48,0.48}{##1}}}
\@namedef{PY@tok@c1}{\let\PY@it=\textit\def\PY@tc##1{\textcolor[rgb]{0.24,0.48,0.48}{##1}}}
\@namedef{PY@tok@cs}{\let\PY@it=\textit\def\PY@tc##1{\textcolor[rgb]{0.24,0.48,0.48}{##1}}}

\def\PYZbs{\char`\\}
\def\PYZus{\char`\_}
\def\PYZob{\char`\{}
\def\PYZcb{\char`\}}
\def\PYZca{\char`\^}
\def\PYZam{\char`\&}
\def\PYZlt{\char`\<}
\def\PYZgt{\char`\>}
\def\PYZsh{\char`\#}
\def\PYZpc{\char`\%}
\def\PYZdl{\char`\$}
\def\PYZhy{\char`\-}
\def\PYZsq{\char`\'}
\def\PYZdq{\char`\"}
\def\PYZti{\char`\~}
% for compatibility with earlier versions
\def\PYZat{@}
\def\PYZlb{[}
\def\PYZrb{]}
\makeatother


    % For linebreaks inside Verbatim environment from package fancyvrb.
    \makeatletter
        \newbox\Wrappedcontinuationbox
        \newbox\Wrappedvisiblespacebox
        \newcommand*\Wrappedvisiblespace {\textcolor{red}{\textvisiblespace}}
        \newcommand*\Wrappedcontinuationsymbol {\textcolor{red}{\llap{\tiny$\m@th\hookrightarrow$}}}
        \newcommand*\Wrappedcontinuationindent {3ex }
        \newcommand*\Wrappedafterbreak {\kern\Wrappedcontinuationindent\copy\Wrappedcontinuationbox}
        % Take advantage of the already applied Pygments mark-up to insert
        % potential linebreaks for TeX processing.
        %        {, <, #, %, $, ' and ": go to next line.
        %        _, }, ^, &, >, - and ~: stay at end of broken line.
        % Use of \textquotesingle for straight quote.
        \newcommand*\Wrappedbreaksatspecials {%
            \def\PYGZus{\discretionary{\char`\_}{\Wrappedafterbreak}{\char`\_}}%
            \def\PYGZob{\discretionary{}{\Wrappedafterbreak\char`\{}{\char`\{}}%
            \def\PYGZcb{\discretionary{\char`\}}{\Wrappedafterbreak}{\char`\}}}%
            \def\PYGZca{\discretionary{\char`\^}{\Wrappedafterbreak}{\char`\^}}%
            \def\PYGZam{\discretionary{\char`\&}{\Wrappedafterbreak}{\char`\&}}%
            \def\PYGZlt{\discretionary{}{\Wrappedafterbreak\char`\<}{\char`\<}}%
            \def\PYGZgt{\discretionary{\char`\>}{\Wrappedafterbreak}{\char`\>}}%
            \def\PYGZsh{\discretionary{}{\Wrappedafterbreak\char`\#}{\char`\#}}%
            \def\PYGZpc{\discretionary{}{\Wrappedafterbreak\char`\%}{\char`\%}}%
            \def\PYGZdl{\discretionary{}{\Wrappedafterbreak\char`\$}{\char`\$}}%
            \def\PYGZhy{\discretionary{\char`\-}{\Wrappedafterbreak}{\char`\-}}%
            \def\PYGZsq{\discretionary{}{\Wrappedafterbreak\textquotesingle}{\textquotesingle}}%
            \def\PYGZdq{\discretionary{}{\Wrappedafterbreak\char`\"}{\char`\"}}%
            \def\PYGZti{\discretionary{\char`\~}{\Wrappedafterbreak}{\char`\~}}%
        }
        % Some characters . , ; ? ! / are not pygmentized.
        % This macro makes them "active" and they will insert potential linebreaks
        \newcommand*\Wrappedbreaksatpunct {%
            \lccode`\~`\.\lowercase{\def~}{\discretionary{\hbox{\char`\.}}{\Wrappedafterbreak}{\hbox{\char`\.}}}%
            \lccode`\~`\,\lowercase{\def~}{\discretionary{\hbox{\char`\,}}{\Wrappedafterbreak}{\hbox{\char`\,}}}%
            \lccode`\~`\;\lowercase{\def~}{\discretionary{\hbox{\char`\;}}{\Wrappedafterbreak}{\hbox{\char`\;}}}%
            \lccode`\~`\:\lowercase{\def~}{\discretionary{\hbox{\char`\:}}{\Wrappedafterbreak}{\hbox{\char`\:}}}%
            \lccode`\~`\?\lowercase{\def~}{\discretionary{\hbox{\char`\?}}{\Wrappedafterbreak}{\hbox{\char`\?}}}%
            \lccode`\~`\!\lowercase{\def~}{\discretionary{\hbox{\char`\!}}{\Wrappedafterbreak}{\hbox{\char`\!}}}%
            \lccode`\~`\/\lowercase{\def~}{\discretionary{\hbox{\char`\/}}{\Wrappedafterbreak}{\hbox{\char`\/}}}%
            \catcode`\.\active
            \catcode`\,\active
            \catcode`\;\active
            \catcode`\:\active
            \catcode`\?\active
            \catcode`\!\active
            \catcode`\/\active
            \lccode`\~`\~
        }
    \makeatother

    \let\OriginalVerbatim=\Verbatim
    \makeatletter
    \renewcommand{\Verbatim}[1][1]{%
        %\parskip\z@skip
        \sbox\Wrappedcontinuationbox {\Wrappedcontinuationsymbol}%
        \sbox\Wrappedvisiblespacebox {\FV@SetupFont\Wrappedvisiblespace}%
        \def\FancyVerbFormatLine ##1{\hsize\linewidth
            \vtop{\raggedright\hyphenpenalty\z@\exhyphenpenalty\z@
                \doublehyphendemerits\z@\finalhyphendemerits\z@
                \strut ##1\strut}%
        }%
        % If the linebreak is at a space, the latter will be displayed as visible
        % space at end of first line, and a continuation symbol starts next line.
        % Stretch/shrink are however usually zero for typewriter font.
        \def\FV@Space {%
            \nobreak\hskip\z@ plus\fontdimen3\font minus\fontdimen4\font
            \discretionary{\copy\Wrappedvisiblespacebox}{\Wrappedafterbreak}
            {\kern\fontdimen2\font}%
        }%

        % Allow breaks at special characters using \PYG... macros.
        \Wrappedbreaksatspecials
        % Breaks at punctuation characters . , ; ? ! and / need catcode=\active
        \OriginalVerbatim[#1,codes*=\Wrappedbreaksatpunct]%
    }
    \makeatother

    % Exact colors from NB
    \definecolor{incolor}{HTML}{303F9F}
    \definecolor{outcolor}{HTML}{D84315}
    \definecolor{cellborder}{HTML}{CFCFCF}
    \definecolor{cellbackground}{HTML}{F7F7F7}

    % prompt
    \makeatletter
    \newcommand{\boxspacing}{\kern\kvtcb@left@rule\kern\kvtcb@boxsep}
    \makeatother
    \newcommand{\prompt}[4]{
        {\ttfamily\llap{{\color{#2}[#3]:\hspace{3pt}#4}}\vspace{-\baselineskip}}
    }
    

    
    % Prevent overflowing lines due to hard-to-break entities
    \sloppy
    % Setup hyperref package
    \hypersetup{
      breaklinks=true,  % so long urls are correctly broken across lines
      colorlinks=true,
      urlcolor=urlcolor,
      linkcolor=linkcolor,
      citecolor=citecolor,
      }
    % Slightly bigger margins than the latex defaults
    
    \geometry{verbose,tmargin=1in,bmargin=1in,lmargin=1in,rmargin=1in}
    
    

\begin{document}
    
    \maketitle
    
    

    
    \begin{tcolorbox}[breakable, size=fbox, boxrule=1pt, pad at break*=1mm,colback=cellbackground, colframe=cellborder]
\prompt{In}{incolor}{1}{\boxspacing}
\begin{Verbatim}[commandchars=\\\{\}]
\PY{k+kn}{import} \PY{n+nn}{pandas} \PY{k}{as} \PY{n+nn}{pd}
\PY{k+kn}{import} \PY{n+nn}{seaborn} \PY{k}{as} \PY{n+nn}{sns}
\PY{k+kn}{import} \PY{n+nn}{matplotlib}\PY{n+nn}{.}\PY{n+nn}{pyplot} \PY{k}{as} \PY{n+nn}{plt}
\PY{k+kn}{import} \PY{n+nn}{statsmodels}\PY{n+nn}{.}\PY{n+nn}{api} \PY{k}{as} \PY{n+nn}{sm}
\PY{k+kn}{from} \PY{n+nn}{sklearn}\PY{n+nn}{.}\PY{n+nn}{metrics} \PY{k+kn}{import} \PY{n}{mean\PYZus{}squared\PYZus{}error}\PY{p}{,} \PY{n}{r2\PYZus{}score}
\end{Verbatim}
\end{tcolorbox}

    \hypertarget{national-employment-and-gdp}{%
\section{1. National Employment and
GDP}\label{national-employment-and-gdp}}

Initially, We did some EDA on the provided data. Following are the codes
and findings.

    \begin{tcolorbox}[breakable, size=fbox, boxrule=1pt, pad at break*=1mm,colback=cellbackground, colframe=cellborder]
\prompt{In}{incolor}{2}{\boxspacing}
\begin{Verbatim}[commandchars=\\\{\}]
\PY{n}{employment\PYZus{}gdp\PYZus{}dataframe} \PY{o}{=} \PY{n}{pd}\PY{o}{.}\PY{n}{read\PYZus{}excel}\PY{p}{(}\PY{l+s+s2}{\PYZdq{}}\PY{l+s+s2}{./HW2\PYZus{}Datasets.xlsx}\PY{l+s+s2}{\PYZdq{}}\PY{p}{)}
\PY{n}{employment\PYZus{}gdp\PYZus{}dataframe}\PY{o}{.}\PY{n}{head}\PY{p}{(}\PY{p}{)}
\end{Verbatim}
\end{tcolorbox}

            \begin{tcolorbox}[breakable, size=fbox, boxrule=.5pt, pad at break*=1mm, opacityfill=0]
\prompt{Out}{outcolor}{2}{\boxspacing}
\begin{Verbatim}[commandchars=\\\{\}]
   Quarter  Employment (millions)  GDP (billions)
0  Q1 2020                    155           21500
1  Q2 2020                    152           21200
2  Q3 2020                    150           21700
3  Q4 2020                    153           22000
4  Q1 2021                    154           22200
\end{Verbatim}
\end{tcolorbox}
        
    \begin{tcolorbox}[breakable, size=fbox, boxrule=1pt, pad at break*=1mm,colback=cellbackground, colframe=cellborder]
\prompt{In}{incolor}{3}{\boxspacing}
\begin{Verbatim}[commandchars=\\\{\}]
\PY{n}{employment\PYZus{}gdp\PYZus{}dataframe}\PY{o}{.}\PY{n}{info}\PY{p}{(}\PY{p}{)}
\end{Verbatim}
\end{tcolorbox}

    \begin{Verbatim}[commandchars=\\\{\}]
<class 'pandas.core.frame.DataFrame'>
RangeIndex: 20 entries, 0 to 19
Data columns (total 3 columns):
 \#   Column                 Non-Null Count  Dtype
---  ------                 --------------  -----
 0   Quarter                20 non-null     object
 1   Employment (millions)  20 non-null     int64
 2   GDP (billions)         20 non-null     int64
dtypes: int64(2), object(1)
memory usage: 612.0+ bytes
    \end{Verbatim}

    \begin{tcolorbox}[breakable, size=fbox, boxrule=1pt, pad at break*=1mm,colback=cellbackground, colframe=cellborder]
\prompt{In}{incolor}{4}{\boxspacing}
\begin{Verbatim}[commandchars=\\\{\}]
\PY{n}{employment\PYZus{}gdp\PYZus{}dataframe}\PY{o}{.}\PY{n}{describe}\PY{p}{(}\PY{p}{)}
\end{Verbatim}
\end{tcolorbox}

            \begin{tcolorbox}[breakable, size=fbox, boxrule=.5pt, pad at break*=1mm, opacityfill=0]
\prompt{Out}{outcolor}{4}{\boxspacing}
\begin{Verbatim}[commandchars=\\\{\}]
       Employment (millions)  GDP (billions)
count              20.000000       20.000000
mean              164.150000    23437.500000
std                 9.669676     1385.818683
min               150.000000    21200.000000
25\%               155.750000    22350.000000
50\%               163.000000    23400.000000
75\%               171.500000    24550.000000
max               181.000000    25750.000000
\end{Verbatim}
\end{tcolorbox}
        
    \begin{tcolorbox}[breakable, size=fbox, boxrule=1pt, pad at break*=1mm,colback=cellbackground, colframe=cellborder]
\prompt{In}{incolor}{5}{\boxspacing}
\begin{Verbatim}[commandchars=\\\{\}]
\PY{c+c1}{\PYZsh{} Splitting Quarter field into years, quarter number.}
\PY{n}{employment\PYZus{}gdp\PYZus{}dataframe}\PY{p}{[}\PY{l+s+s2}{\PYZdq{}}\PY{l+s+s2}{Year}\PY{l+s+s2}{\PYZdq{}}\PY{p}{]} \PY{o}{=} \PY{n}{employment\PYZus{}gdp\PYZus{}dataframe}\PY{p}{[}\PY{l+s+s2}{\PYZdq{}}\PY{l+s+s2}{Quarter}\PY{l+s+s2}{\PYZdq{}}\PY{p}{]}\PY{o}{.}\PY{n}{str}\PY{p}{[}\PY{o}{\PYZhy{}}\PY{l+m+mi}{4}\PY{p}{:}\PY{p}{]}\PY{o}{.}\PY{n}{astype}\PY{p}{(}\PY{n+nb}{int}\PY{p}{)}
\PY{n}{employment\PYZus{}gdp\PYZus{}dataframe}\PY{p}{[}\PY{l+s+s1}{\PYZsq{}}\PY{l+s+s1}{Quarter\PYZus{}num}\PY{l+s+s1}{\PYZsq{}}\PY{p}{]} \PY{o}{=} \PY{n}{employment\PYZus{}gdp\PYZus{}dataframe}\PY{p}{[}\PY{l+s+s1}{\PYZsq{}}\PY{l+s+s1}{Quarter}\PY{l+s+s1}{\PYZsq{}}\PY{p}{]}\PY{o}{.}\PY{n}{str}\PY{p}{[}\PY{l+m+mi}{1}\PY{p}{]}\PY{o}{.}\PY{n}{astype}\PY{p}{(}\PY{n+nb}{int}\PY{p}{)}
\PY{n}{employment\PYZus{}gdp\PYZus{}dataframe}\PY{p}{[}\PY{l+s+s2}{\PYZdq{}}\PY{l+s+s2}{Lagged Employment}\PY{l+s+s2}{\PYZdq{}}\PY{p}{]} \PY{o}{=} \PY{n}{employment\PYZus{}gdp\PYZus{}dataframe}\PY{p}{[}\PY{l+s+s2}{\PYZdq{}}\PY{l+s+s2}{Employment (millions)}\PY{l+s+s2}{\PYZdq{}}\PY{p}{]}\PY{o}{.}\PY{n}{shift}\PY{p}{(}\PY{l+m+mi}{1}\PY{p}{)}
\PY{n}{employment\PYZus{}gdp\PYZus{}dataframe} \PY{o}{=} \PY{n}{employment\PYZus{}gdp\PYZus{}dataframe}\PY{o}{.}\PY{n}{dropna}\PY{p}{(}\PY{p}{)}
\PY{n}{employment\PYZus{}gdp\PYZus{}dataframe} \PY{o}{=} \PY{n}{employment\PYZus{}gdp\PYZus{}dataframe}\PY{o}{.}\PY{n}{sort\PYZus{}values}\PY{p}{(}\PY{n}{by}\PY{o}{=} \PY{p}{[}\PY{l+s+s2}{\PYZdq{}}\PY{l+s+s2}{Year}\PY{l+s+s2}{\PYZdq{}}\PY{p}{,} \PY{l+s+s2}{\PYZdq{}}\PY{l+s+s2}{Quarter\PYZus{}num}\PY{l+s+s2}{\PYZdq{}}\PY{p}{]}\PY{p}{)}\PY{o}{.}\PY{n}{reset\PYZus{}index}\PY{p}{(}\PY{p}{)}\PY{o}{.}\PY{n}{drop}\PY{p}{(}\PY{n}{columns}\PY{o}{=} \PY{l+s+s2}{\PYZdq{}}\PY{l+s+s2}{index}\PY{l+s+s2}{\PYZdq{}}\PY{p}{)}
\PY{n}{employment\PYZus{}gdp\PYZus{}dataframe} \PY{o}{=} \PY{n}{employment\PYZus{}gdp\PYZus{}dataframe}\PY{o}{.}\PY{n}{reset\PYZus{}index}\PY{p}{(}\PY{p}{)}\PY{o}{.}\PY{n}{rename}\PY{p}{(}\PY{n}{columns}\PY{o}{=}\PY{p}{\PYZob{}}\PY{l+s+s1}{\PYZsq{}}\PY{l+s+s1}{index}\PY{l+s+s1}{\PYZsq{}}\PY{p}{:} \PY{l+s+s1}{\PYZsq{}}\PY{l+s+s1}{Coded Time}\PY{l+s+s1}{\PYZsq{}}\PY{p}{\PYZcb{}}\PY{p}{)}
\end{Verbatim}
\end{tcolorbox}

    \begin{tcolorbox}[breakable, size=fbox, boxrule=1pt, pad at break*=1mm,colback=cellbackground, colframe=cellborder]
\prompt{In}{incolor}{6}{\boxspacing}
\begin{Verbatim}[commandchars=\\\{\}]
\PY{c+c1}{\PYZsh{}OnehotEncoding}
\PY{k+kn}{from} \PY{n+nn}{sklearn}\PY{n+nn}{.}\PY{n+nn}{preprocessing} \PY{k+kn}{import} \PY{n}{OneHotEncoder}
\PY{n}{encoder} \PY{o}{=} \PY{n}{OneHotEncoder}\PY{p}{(}\PY{n}{sparse\PYZus{}output}\PY{o}{=} \PY{k+kc}{False}\PY{p}{,} \PY{n}{drop}\PY{o}{=} \PY{l+s+s2}{\PYZdq{}}\PY{l+s+s2}{first}\PY{l+s+s2}{\PYZdq{}}\PY{p}{)}

\PY{n}{encoded\PYZus{}quarters} \PY{o}{=} \PY{n}{encoder}\PY{o}{.}\PY{n}{fit\PYZus{}transform}\PY{p}{(}\PY{n}{employment\PYZus{}gdp\PYZus{}dataframe}\PY{p}{[}\PY{p}{[}\PY{l+s+s2}{\PYZdq{}}\PY{l+s+s2}{Quarter\PYZus{}num}\PY{l+s+s2}{\PYZdq{}}\PY{p}{]}\PY{p}{]}\PY{p}{)}\PY{o}{.}\PY{n}{astype}\PY{p}{(}\PY{n+nb}{int}\PY{p}{)}
\PY{n}{df\PYZus{}encoded\PYZus{}quarters} \PY{o}{=} \PY{n}{pd}\PY{o}{.}\PY{n}{DataFrame}\PY{p}{(}\PY{n}{data}\PY{o}{=} \PY{n}{encoded\PYZus{}quarters}\PY{p}{,} \PY{n}{columns}\PY{o}{=} \PY{n}{encoder}\PY{o}{.}\PY{n}{get\PYZus{}feature\PYZus{}names\PYZus{}out}\PY{p}{(}\PY{p}{)}\PY{p}{)}
\PY{n}{df\PYZus{}encoded} \PY{o}{=} \PY{n}{pd}\PY{o}{.}\PY{n}{concat}\PY{p}{(}\PY{p}{[}\PY{n}{employment\PYZus{}gdp\PYZus{}dataframe}\PY{p}{,} \PY{n}{df\PYZus{}encoded\PYZus{}quarters}\PY{p}{]}\PY{p}{,} \PY{n}{axis}\PY{o}{=} \PY{l+m+mi}{1}\PY{p}{)}
\PY{n}{df\PYZus{}encoded}\PY{o}{.}\PY{n}{head}\PY{p}{(}\PY{p}{)}
\end{Verbatim}
\end{tcolorbox}

            \begin{tcolorbox}[breakable, size=fbox, boxrule=.5pt, pad at break*=1mm, opacityfill=0]
\prompt{Out}{outcolor}{6}{\boxspacing}
\begin{Verbatim}[commandchars=\\\{\}]
   Coded Time  Quarter  Employment (millions)  GDP (billions)  Year  \textbackslash{}
0           0  Q2 2020                    152           21200  2020
1           1  Q3 2020                    150           21700  2020
2           2  Q4 2020                    153           22000  2020
3           3  Q1 2021                    154           22200  2021
4           4  Q2 2021                    156           22400  2021

   Quarter\_num  Lagged Employment  Quarter\_num\_2  Quarter\_num\_3  Quarter\_num\_4
0            2              155.0              1              0              0
1            3              152.0              0              1              0
2            4              150.0              0              0              1
3            1              153.0              0              0              0
4            2              154.0              1              0              0
\end{Verbatim}
\end{tcolorbox}
        
    \begin{tcolorbox}[breakable, size=fbox, boxrule=1pt, pad at break*=1mm,colback=cellbackground, colframe=cellborder]
\prompt{In}{incolor}{7}{\boxspacing}
\begin{Verbatim}[commandchars=\\\{\}]
\PY{n}{df\PYZus{}encoded}\PY{o}{.}\PY{n}{info}\PY{p}{(}\PY{p}{)}
\end{Verbatim}
\end{tcolorbox}

    \begin{Verbatim}[commandchars=\\\{\}]
<class 'pandas.core.frame.DataFrame'>
RangeIndex: 19 entries, 0 to 18
Data columns (total 10 columns):
 \#   Column                 Non-Null Count  Dtype
---  ------                 --------------  -----
 0   Coded Time             19 non-null     int64
 1   Quarter                19 non-null     object
 2   Employment (millions)  19 non-null     int64
 3   GDP (billions)         19 non-null     int64
 4   Year                   19 non-null     int64
 5   Quarter\_num            19 non-null     int64
 6   Lagged Employment      19 non-null     float64
 7   Quarter\_num\_2          19 non-null     int64
 8   Quarter\_num\_3          19 non-null     int64
 9   Quarter\_num\_4          19 non-null     int64
dtypes: float64(1), int64(8), object(1)
memory usage: 1.6+ KB
    \end{Verbatim}

    \begin{tcolorbox}[breakable, size=fbox, boxrule=1pt, pad at break*=1mm,colback=cellbackground, colframe=cellborder]
\prompt{In}{incolor}{8}{\boxspacing}
\begin{Verbatim}[commandchars=\\\{\}]
\PY{c+c1}{\PYZsh{} Multivariate linear regression model}
\PY{n}{X\PYZus{}train} \PY{o}{=} \PY{n}{df\PYZus{}encoded}\PY{o}{.}\PY{n}{drop}\PY{p}{(}\PY{n}{columns}\PY{o}{=} \PY{p}{[}\PY{l+s+s2}{\PYZdq{}}\PY{l+s+s2}{Quarter}\PY{l+s+s2}{\PYZdq{}}\PY{p}{,} \PY{l+s+s2}{\PYZdq{}}\PY{l+s+s2}{GDP (billions)}\PY{l+s+s2}{\PYZdq{}}\PY{p}{,} \PY{l+s+s2}{\PYZdq{}}\PY{l+s+s2}{Employment (millions)}\PY{l+s+s2}{\PYZdq{}}\PY{p}{,} \PY{l+s+s2}{\PYZdq{}}\PY{l+s+s2}{Quarter\PYZus{}num}\PY{l+s+s2}{\PYZdq{}}\PY{p}{,} \PY{l+s+s2}{\PYZdq{}}\PY{l+s+s2}{Coded Time}\PY{l+s+s2}{\PYZdq{}}\PY{p}{]}\PY{p}{)}
\PY{n}{y\PYZus{}train} \PY{o}{=} \PY{n}{df\PYZus{}encoded}\PY{p}{[}\PY{p}{[}\PY{l+s+s2}{\PYZdq{}}\PY{l+s+s2}{GDP (billions)}\PY{l+s+s2}{\PYZdq{}}\PY{p}{]}\PY{p}{]}
\PY{n}{X\PYZus{}train\PYZus{}with\PYZus{}const} \PY{o}{=} \PY{n}{sm}\PY{o}{.}\PY{n}{add\PYZus{}constant}\PY{p}{(}\PY{n}{X\PYZus{}train}\PY{p}{)}

\PY{n}{multivariate\PYZus{}ols\PYZus{}model} \PY{o}{=} \PY{n}{sm}\PY{o}{.}\PY{n}{OLS}\PY{p}{(}\PY{n}{y\PYZus{}train}\PY{p}{,} \PY{n}{X\PYZus{}train\PYZus{}with\PYZus{}const}\PY{p}{)}\PY{o}{.}\PY{n}{fit}\PY{p}{(}\PY{p}{)}
\end{Verbatim}
\end{tcolorbox}

    \begin{tcolorbox}[breakable, size=fbox, boxrule=1pt, pad at break*=1mm,colback=cellbackground, colframe=cellborder]
\prompt{In}{incolor}{9}{\boxspacing}
\begin{Verbatim}[commandchars=\\\{\}]
\PY{n}{y\PYZus{}predicted\PYZus{}multivariiate} \PY{o}{=} \PY{n}{multivariate\PYZus{}ols\PYZus{}model}\PY{o}{.}\PY{n}{predict}\PY{p}{(}\PY{n}{X\PYZus{}train\PYZus{}with\PYZus{}const}\PY{p}{)}
\PY{n+nb}{print}\PY{p}{(}\PY{l+s+sa}{f}\PY{l+s+s2}{\PYZdq{}}\PY{l+s+s2}{Mean\PYZus{}squared error for multivariate linear regression: }\PY{l+s+si}{\PYZob{}}\PY{n}{mean\PYZus{}squared\PYZus{}error}\PY{p}{(}\PY{n}{y\PYZus{}true}\PY{o}{=}\PY{+w}{ }\PY{n}{y\PYZus{}train}\PY{p}{,}\PY{+w}{ }\PY{n}{y\PYZus{}pred}\PY{o}{=}\PY{+w}{ }\PY{n}{y\PYZus{}predicted\PYZus{}multivariiate}\PY{p}{)}\PY{l+s+si}{\PYZcb{}}\PY{l+s+s2}{\PYZdq{}}\PY{p}{)}
\end{Verbatim}
\end{tcolorbox}

    \begin{Verbatim}[commandchars=\\\{\}]
Mean\_squared error for multivariate linear regression: 2961.873985624827
    \end{Verbatim}

    \begin{tcolorbox}[breakable, size=fbox, boxrule=1pt, pad at break*=1mm,colback=cellbackground, colframe=cellborder]
\prompt{In}{incolor}{10}{\boxspacing}
\begin{Verbatim}[commandchars=\\\{\}]
\PY{c+c1}{\PYZsh{}Univariate linear regression model}
\PY{n}{X\PYZus{}train\PYZus{}univariate} \PY{o}{=} \PY{n}{df\PYZus{}encoded}\PY{p}{[}\PY{p}{[}\PY{l+s+s2}{\PYZdq{}}\PY{l+s+s2}{Lagged Employment}\PY{l+s+s2}{\PYZdq{}}\PY{p}{]}\PY{p}{]}

\PY{n}{X\PYZus{}train\PYZus{}univariate\PYZus{}with\PYZus{}const} \PY{o}{=} \PY{n}{sm}\PY{o}{.}\PY{n}{add\PYZus{}constant}\PY{p}{(}\PY{n}{X\PYZus{}train\PYZus{}univariate}\PY{p}{)}

\PY{n}{univariate\PYZus{}ols\PYZus{}model} \PY{o}{=} \PY{n}{sm}\PY{o}{.}\PY{n}{OLS}\PY{p}{(}\PY{n}{y\PYZus{}train}\PY{p}{,} \PY{n}{X\PYZus{}train\PYZus{}univariate\PYZus{}with\PYZus{}const}\PY{p}{)}\PY{o}{.}\PY{n}{fit}\PY{p}{(}\PY{p}{)}
\end{Verbatim}
\end{tcolorbox}

    \begin{tcolorbox}[breakable, size=fbox, boxrule=1pt, pad at break*=1mm,colback=cellbackground, colframe=cellborder]
\prompt{In}{incolor}{11}{\boxspacing}
\begin{Verbatim}[commandchars=\\\{\}]
\PY{n}{y\PYZus{}predicted\PYZus{}univariiate} \PY{o}{=} \PY{n}{univariate\PYZus{}ols\PYZus{}model}\PY{o}{.}\PY{n}{predict}\PY{p}{(}\PY{n}{X\PYZus{}train\PYZus{}univariate\PYZus{}with\PYZus{}const}\PY{p}{)}
\PY{n+nb}{print}\PY{p}{(}\PY{l+s+sa}{f}\PY{l+s+s2}{\PYZdq{}}\PY{l+s+s2}{Mean\PYZus{}squared error for univariate linear regression: }\PY{l+s+si}{\PYZob{}}\PY{n}{mean\PYZus{}squared\PYZus{}error}\PY{p}{(}\PY{n}{y\PYZus{}true}\PY{o}{=}\PY{+w}{ }\PY{n}{y\PYZus{}train}\PY{p}{,}\PY{+w}{ }\PY{n}{y\PYZus{}pred}\PY{o}{=}\PY{+w}{ }\PY{n}{y\PYZus{}predicted\PYZus{}univariiate}\PY{p}{)}\PY{l+s+si}{\PYZcb{}}\PY{l+s+s2}{\PYZdq{}}\PY{p}{)}
\end{Verbatim}
\end{tcolorbox}

    \begin{Verbatim}[commandchars=\\\{\}]
Mean\_squared error for univariate linear regression: 88189.44444027875
    \end{Verbatim}

    \hypertarget{residual-scatterplot-create-a-scatterplot-of-the-residuals-against-the-lagged-employment.}{%
\subsubsection{Residual Scatterplot: Create a scatterplot of the
residuals against the lagged
employment.}\label{residual-scatterplot-create-a-scatterplot-of-the-residuals-against-the-lagged-employment.}}

=\textgreater{} We created the residual scatter plots using multivariate
linear regression and univariate linear regression. In multivariate
linear regression, we used Year, Lagged Employment and one hot encoded
quarter numbers. For Univariate linear regression, We only used Lagged
Employment as the predictor and ``GDP (Billions)'' as dependent
variable.

    \begin{tcolorbox}[breakable, size=fbox, boxrule=1pt, pad at break*=1mm,colback=cellbackground, colframe=cellborder]
\prompt{In}{incolor}{12}{\boxspacing}
\begin{Verbatim}[commandchars=\\\{\}]
\PY{n}{fig}\PY{p}{,} \PY{n}{axs} \PY{o}{=} \PY{n}{plt}\PY{o}{.}\PY{n}{subplots}\PY{p}{(}\PY{l+m+mi}{1}\PY{p}{,} \PY{l+m+mi}{2}\PY{p}{,} \PY{n}{figsize}\PY{o}{=} \PY{p}{(}\PY{l+m+mi}{15}\PY{p}{,} \PY{l+m+mi}{5}\PY{p}{)}\PY{p}{)}

\PY{n}{sns}\PY{o}{.}\PY{n}{residplot}\PY{p}{(}\PY{n}{x} \PY{o}{=} \PY{n}{y\PYZus{}predicted\PYZus{}multivariiate}\PY{p}{,} \PY{n}{y} \PY{o}{=} \PY{n}{multivariate\PYZus{}ols\PYZus{}model}\PY{o}{.}\PY{n}{resid}\PY{p}{,} \PY{n}{lowess}\PY{o}{=}\PY{k+kc}{True}\PY{p}{,} \PY{n}{color}\PY{o}{=}\PY{l+s+s1}{\PYZsq{}}\PY{l+s+s1}{blue}\PY{l+s+s1}{\PYZsq{}}\PY{p}{,} \PY{n}{ax}\PY{o}{=} \PY{n}{axs}\PY{p}{[}\PY{l+m+mi}{0}\PY{p}{]}\PY{p}{)}
\PY{n}{axs}\PY{p}{[}\PY{l+m+mi}{0}\PY{p}{]}\PY{o}{.}\PY{n}{axhline}\PY{p}{(}\PY{l+m+mi}{0}\PY{p}{,} \PY{n}{color}\PY{o}{=}\PY{l+s+s1}{\PYZsq{}}\PY{l+s+s1}{red}\PY{l+s+s1}{\PYZsq{}}\PY{p}{,} \PY{n}{linestyle}\PY{o}{=}\PY{l+s+s1}{\PYZsq{}}\PY{l+s+s1}{\PYZhy{}\PYZhy{}}\PY{l+s+s1}{\PYZsq{}}\PY{p}{)}
\PY{n}{axs}\PY{p}{[}\PY{l+m+mi}{0}\PY{p}{]}\PY{o}{.}\PY{n}{set\PYZus{}xlabel}\PY{p}{(}\PY{l+s+s2}{\PYZdq{}}\PY{l+s+s2}{Predicted GDP (Billion)}\PY{l+s+s2}{\PYZdq{}}\PY{p}{)}
\PY{n}{axs}\PY{p}{[}\PY{l+m+mi}{0}\PY{p}{]}\PY{o}{.}\PY{n}{set\PYZus{}ylabel}\PY{p}{(}\PY{l+s+s2}{\PYZdq{}}\PY{l+s+s2}{Residual error between predicted and original GDP}\PY{l+s+s2}{\PYZdq{}}\PY{p}{)}
\PY{n}{axs}\PY{p}{[}\PY{l+m+mi}{0}\PY{p}{]}\PY{o}{.}\PY{n}{set\PYZus{}title}\PY{p}{(}\PY{l+s+s2}{\PYZdq{}}\PY{l+s+s2}{Residual plot for multivariate regression model}\PY{l+s+s2}{\PYZdq{}}\PY{p}{,} \PY{n}{fontsize} \PY{o}{=} \PY{l+m+mi}{15}\PY{p}{)}

\PY{n}{sns}\PY{o}{.}\PY{n}{residplot}\PY{p}{(}\PY{n}{x} \PY{o}{=} \PY{n}{y\PYZus{}predicted\PYZus{}univariiate}\PY{p}{,} \PY{n}{y} \PY{o}{=} \PY{n}{univariate\PYZus{}ols\PYZus{}model}\PY{o}{.}\PY{n}{resid}\PY{p}{,} \PY{n}{lowess}\PY{o}{=}\PY{k+kc}{True}\PY{p}{,} \PY{n}{color}\PY{o}{=}\PY{l+s+s1}{\PYZsq{}}\PY{l+s+s1}{blue}\PY{l+s+s1}{\PYZsq{}}\PY{p}{,} \PY{n}{ax}\PY{o}{=} \PY{n}{axs}\PY{p}{[}\PY{l+m+mi}{1}\PY{p}{]}\PY{p}{)}
\PY{n}{axs}\PY{p}{[}\PY{l+m+mi}{1}\PY{p}{]}\PY{o}{.}\PY{n}{axhline}\PY{p}{(}\PY{l+m+mi}{0}\PY{p}{,} \PY{n}{color}\PY{o}{=}\PY{l+s+s1}{\PYZsq{}}\PY{l+s+s1}{green}\PY{l+s+s1}{\PYZsq{}}\PY{p}{,} \PY{n}{linestyle}\PY{o}{=}\PY{l+s+s1}{\PYZsq{}}\PY{l+s+s1}{\PYZhy{}\PYZhy{}}\PY{l+s+s1}{\PYZsq{}}\PY{p}{)}
\PY{n}{axs}\PY{p}{[}\PY{l+m+mi}{1}\PY{p}{]}\PY{o}{.}\PY{n}{set\PYZus{}xlabel}\PY{p}{(}\PY{l+s+s2}{\PYZdq{}}\PY{l+s+s2}{Predicted GDP (Billion)}\PY{l+s+s2}{\PYZdq{}}\PY{p}{)}
\PY{n}{axs}\PY{p}{[}\PY{l+m+mi}{1}\PY{p}{]}\PY{o}{.}\PY{n}{set\PYZus{}ylabel}\PY{p}{(}\PY{l+s+s2}{\PYZdq{}}\PY{l+s+s2}{Residual error between predicted and original GDP}\PY{l+s+s2}{\PYZdq{}}\PY{p}{)}
\PY{n}{axs}\PY{p}{[}\PY{l+m+mi}{1}\PY{p}{]}\PY{o}{.}\PY{n}{set\PYZus{}title}\PY{p}{(}\PY{l+s+s2}{\PYZdq{}}\PY{l+s+s2}{Residual plot for univariate regression model}\PY{l+s+s2}{\PYZdq{}}\PY{p}{,} \PY{n}{fontsize} \PY{o}{=} \PY{l+m+mi}{15}\PY{p}{)}

\PY{n}{plt}\PY{o}{.}\PY{n}{tight\PYZus{}layout}\PY{p}{(}\PY{p}{)}
\PY{n}{plt}\PY{o}{.}\PY{n}{show}\PY{p}{(}\PY{p}{)}
\end{Verbatim}
\end{tcolorbox}

    \begin{center}
    \adjustimage{max size={0.9\linewidth}{0.9\paperheight}}{E923Q669_assignment2_files/E923Q669_assignment2_13_0.png}
    \end{center}
    { \hspace*{\fill} \\}
    
    \begin{tcolorbox}[breakable, size=fbox, boxrule=1pt, pad at break*=1mm,colback=cellbackground, colframe=cellborder]
\prompt{In}{incolor}{13}{\boxspacing}
\begin{Verbatim}[commandchars=\\\{\}]
\PY{n}{plt}\PY{o}{.}\PY{n}{figure}\PY{p}{(}\PY{n}{figsize}\PY{o}{=}\PY{p}{(}\PY{l+m+mi}{7}\PY{p}{,} \PY{l+m+mi}{4}\PY{p}{)}\PY{p}{)}
\PY{n}{sns}\PY{o}{.}\PY{n}{residplot}\PY{p}{(}\PY{n}{y} \PY{o}{=} \PY{l+s+s2}{\PYZdq{}}\PY{l+s+s2}{GDP (billions)}\PY{l+s+s2}{\PYZdq{}}\PY{p}{,} \PY{n}{x} \PY{o}{=} \PY{l+s+s2}{\PYZdq{}}\PY{l+s+s2}{Lagged Employment}\PY{l+s+s2}{\PYZdq{}}\PY{p}{,} \PY{n}{data}\PY{o}{=} \PY{n}{df\PYZus{}encoded}\PY{p}{)}
\PY{n}{plt}\PY{o}{.}\PY{n}{title}\PY{p}{(}\PY{l+s+s2}{\PYZdq{}}\PY{l+s+s2}{Residual plot between Lagged Employment and GDP (billions)}\PY{l+s+s2}{\PYZdq{}}\PY{p}{)}
\PY{n}{plt}\PY{o}{.}\PY{n}{tight\PYZus{}layout}\PY{p}{(}\PY{p}{)}
\PY{n}{plt}\PY{o}{.}\PY{n}{show}\PY{p}{(}\PY{p}{)}
\end{Verbatim}
\end{tcolorbox}

    \begin{center}
    \adjustimage{max size={0.9\linewidth}{0.9\paperheight}}{E923Q669_assignment2_files/E923Q669_assignment2_14_0.png}
    \end{center}
    { \hspace*{\fill} \\}
    
    \hypertarget{does-the-plot-suggest-that-the-residuals-are-homoscedastic}{%
\subsubsection{Does the plot suggest that the residuals are
homoscedastic?}\label{does-the-plot-suggest-that-the-residuals-are-homoscedastic}}

As we can see the residual plots for both univariate and multivariate
linear regressions, residual plot does not lie within the specific width
around the x-axis. Residuals are randomly scattered around the
horizontal axis with no clear pattern. Hence, the residuals are not
homoscedastic for both univariate and multivariate linear regressions.

    \hypertarget{residual-histogram-create-a-histogram-of-the-residuals-and-compare-it-to-the-normal-distribution}{%
\subsubsection{Residual Histogram: Create a histogram of the residuals
and compare it to the normal
distribution}\label{residual-histogram-create-a-histogram-of-the-residuals-and-compare-it-to-the-normal-distribution}}

=\textgreater{} We plotted the histogram for the residuals from both
univariate linear regression and multivariate linear regression. Below
is the code and plot generated.

    \begin{tcolorbox}[breakable, size=fbox, boxrule=1pt, pad at break*=1mm,colback=cellbackground, colframe=cellborder]
\prompt{In}{incolor}{14}{\boxspacing}
\begin{Verbatim}[commandchars=\\\{\}]
\PY{n}{fig}\PY{p}{,} \PY{n}{axs}\PY{o}{=} \PY{n}{plt}\PY{o}{.}\PY{n}{subplots}\PY{p}{(}\PY{l+m+mi}{1}\PY{p}{,} \PY{l+m+mi}{2}\PY{p}{,} \PY{n}{figsize}\PY{o}{=} \PY{p}{(}\PY{l+m+mi}{15}\PY{p}{,} \PY{l+m+mi}{5}\PY{p}{)}\PY{p}{)}

\PY{n}{sns}\PY{o}{.}\PY{n}{histplot}\PY{p}{(}\PY{n}{data}\PY{o}{=} \PY{n}{multivariate\PYZus{}ols\PYZus{}model}\PY{o}{.}\PY{n}{resid}\PY{p}{,}  \PY{n}{kde}\PY{o}{=} \PY{k+kc}{True}\PY{p}{,}\PY{n}{bins}\PY{o}{=}\PY{l+m+mi}{10}\PY{p}{,} \PY{n}{color}\PY{o}{=}\PY{l+s+s1}{\PYZsq{}}\PY{l+s+s1}{purple}\PY{l+s+s1}{\PYZsq{}}\PY{p}{,} \PY{n}{ax}\PY{o}{=} \PY{n}{axs}\PY{p}{[}\PY{l+m+mi}{0}\PY{p}{]}\PY{p}{)}
\PY{n}{axs}\PY{p}{[}\PY{l+m+mi}{0}\PY{p}{]}\PY{o}{.}\PY{n}{set\PYZus{}xlabel}\PY{p}{(}\PY{l+s+s2}{\PYZdq{}}\PY{l+s+s2}{Residuals (Multivariate linear regression)}\PY{l+s+s2}{\PYZdq{}}\PY{p}{,} \PY{n}{fontsize} \PY{o}{=} \PY{l+m+mi}{10}\PY{p}{)}
\PY{n}{axs}\PY{p}{[}\PY{l+m+mi}{0}\PY{p}{]}\PY{o}{.}\PY{n}{set\PYZus{}title}\PY{p}{(}\PY{l+s+s2}{\PYZdq{}}\PY{l+s+s2}{Histogram plot of residuals for multivariate linear regression}\PY{l+s+s2}{\PYZdq{}}\PY{p}{,} \PY{n}{fontsize} \PY{o}{=} \PY{l+m+mi}{15}\PY{p}{)}

\PY{n}{sns}\PY{o}{.}\PY{n}{histplot}\PY{p}{(}\PY{n}{data}\PY{o}{=} \PY{n}{univariate\PYZus{}ols\PYZus{}model}\PY{o}{.}\PY{n}{resid}\PY{p}{,}  \PY{n}{kde}\PY{o}{=} \PY{k+kc}{True}\PY{p}{,}\PY{n}{bins}\PY{o}{=}\PY{l+m+mi}{10}\PY{p}{,} \PY{n}{color}\PY{o}{=}\PY{l+s+s1}{\PYZsq{}}\PY{l+s+s1}{red}\PY{l+s+s1}{\PYZsq{}}\PY{p}{,} \PY{n}{ax}\PY{o}{=} \PY{n}{axs}\PY{p}{[}\PY{l+m+mi}{1}\PY{p}{]}\PY{p}{)}
\PY{n}{axs}\PY{p}{[}\PY{l+m+mi}{1}\PY{p}{]}\PY{o}{.}\PY{n}{set\PYZus{}xlabel}\PY{p}{(}\PY{l+s+s2}{\PYZdq{}}\PY{l+s+s2}{Residuals (Univariate linear regression)}\PY{l+s+s2}{\PYZdq{}}\PY{p}{,} \PY{n}{fontsize} \PY{o}{=} \PY{l+m+mi}{10}\PY{p}{)}
\PY{n}{axs}\PY{p}{[}\PY{l+m+mi}{1}\PY{p}{]}\PY{o}{.}\PY{n}{set\PYZus{}title}\PY{p}{(}\PY{l+s+s2}{\PYZdq{}}\PY{l+s+s2}{Histogram plot of residuals for Univariate linear regression}\PY{l+s+s2}{\PYZdq{}}\PY{p}{,} \PY{n}{fontsize} \PY{o}{=} \PY{l+m+mi}{15}\PY{p}{)}

\PY{n}{plt}\PY{o}{.}\PY{n}{tight\PYZus{}layout}\PY{p}{(}\PY{p}{)}
\PY{n}{plt}\PY{o}{.}\PY{n}{show}\PY{p}{(}\PY{p}{)}
\end{Verbatim}
\end{tcolorbox}

    \begin{center}
    \adjustimage{max size={0.9\linewidth}{0.9\paperheight}}{E923Q669_assignment2_files/E923Q669_assignment2_17_0.png}
    \end{center}
    { \hspace*{\fill} \\}
    
    \hypertarget{are-the-residuals-approximately-normally-distributed-why-or-why-not}{%
\subsubsection{Are the residuals approximately normally distributed? Why
or why
not?}\label{are-the-residuals-approximately-normally-distributed-why-or-why-not}}

=\textgreater{} No, the residuals from both univariate linear regression
and multivariate linear regression are not normally distributed because
we we can see in figures plotted above, there is shape of both histogram
plots are not resembled with the normal distribution plot.

    \begin{tcolorbox}[breakable, size=fbox, boxrule=1pt, pad at break*=1mm,colback=cellbackground, colframe=cellborder]
\prompt{In}{incolor}{15}{\boxspacing}
\begin{Verbatim}[commandchars=\\\{\}]
\PY{n}{multivariate\PYZus{}ols\PYZus{}model}\PY{o}{.}\PY{n}{summary}\PY{p}{(}\PY{p}{)}
\end{Verbatim}
\end{tcolorbox}

    \begin{Verbatim}[commandchars=\\\{\}]
/home/buddha-thapa-magar/anaconda3/lib/python3.12/site-
packages/scipy/stats/\_axis\_nan\_policy.py:531: UserWarning: kurtosistest only
valid for n>=20 {\ldots} continuing anyway, n=19
  res = hypotest\_fun\_out(*samples, **kwds)
    \end{Verbatim}
 
            
\prompt{Out}{outcolor}{15}{}
    
    \begin{center}
\begin{tabular}{lclc}
\toprule
\textbf{Dep. Variable:}    &  GDP (billions)  & \textbf{  R-squared:         } &     0.998   \\
\textbf{Model:}            &       OLS        & \textbf{  Adj. R-squared:    } &     0.998   \\
\textbf{Method:}           &  Least Squares   & \textbf{  F-statistic:       } &     1501.   \\
\textbf{Date:}             & Sun, 10 Nov 2024 & \textbf{  Prob (F-statistic):} &  1.80e-17   \\
\textbf{Time:}             &     15:19:39     & \textbf{  Log-Likelihood:    } &   -102.90   \\
\textbf{No. Observations:} &          19      & \textbf{  AIC:               } &     217.8   \\
\textbf{Df Residuals:}     &          13      & \textbf{  BIC:               } &     223.5   \\
\textbf{Df Model:}         &           5      & \textbf{                     } &             \\
\textbf{Covariance Type:}  &    nonrobust     & \textbf{                     } &             \\
\bottomrule
\end{tabular}
\begin{tabular}{lcccccc}
                           & \textbf{coef} & \textbf{std err} & \textbf{t} & \textbf{P$> |$t$|$} & \textbf{[0.025} & \textbf{0.975]}  \\
\midrule
\textbf{const}             &   -2.148e+06  &     1.18e+05     &   -18.185  &         0.000        &     -2.4e+06    &    -1.89e+06     \\
\textbf{Year}              &    1075.1900  &       59.122     &    18.186  &         0.000        &      947.465    &     1202.915     \\
\textbf{Lagged Employment} &     -19.3557  &        9.134     &    -2.119  &         0.054        &      -39.089    &        0.377     \\
\textbf{Quarter\_num\_2}   &     253.3205  &       51.444     &     4.924  &         0.000        &      142.183    &      364.458     \\
\textbf{Quarter\_num\_3}   &     572.6762  &       56.585     &    10.121  &         0.000        &      450.432    &      694.921     \\
\textbf{Quarter\_num\_4}   &     832.0319  &       62.643     &    13.282  &         0.000        &      696.700    &      967.364     \\
\bottomrule
\end{tabular}
\begin{tabular}{lclc}
\textbf{Omnibus:}       &  3.870 & \textbf{  Durbin-Watson:     } &    0.790  \\
\textbf{Prob(Omnibus):} &  0.144 & \textbf{  Jarque-Bera (JB):  } &    1.443  \\
\textbf{Skew:}          &  0.194 & \textbf{  Prob(JB):          } &    0.486  \\
\textbf{Kurtosis:}      &  1.707 & \textbf{  Cond. No.          } & 1.59e+07  \\
\bottomrule
\end{tabular}
%\caption{OLS Regression Results}
\end{center}

Notes: \newline
 [1] Standard Errors assume that the covariance matrix of the errors is correctly specified. \newline
 [2] The condition number is large, 1.59e+07. This might indicate that there are \newline
 strong multicollinearity or other numerical problems.

    

    \begin{tcolorbox}[breakable, size=fbox, boxrule=1pt, pad at break*=1mm,colback=cellbackground, colframe=cellborder]
\prompt{In}{incolor}{16}{\boxspacing}
\begin{Verbatim}[commandchars=\\\{\}]
\PY{n}{univariate\PYZus{}ols\PYZus{}model}\PY{o}{.}\PY{n}{summary}\PY{p}{(}\PY{p}{)}
\end{Verbatim}
\end{tcolorbox}

    \begin{Verbatim}[commandchars=\\\{\}]
/home/buddha-thapa-magar/anaconda3/lib/python3.12/site-
packages/scipy/stats/\_axis\_nan\_policy.py:531: UserWarning: kurtosistest only
valid for n>=20 {\ldots} continuing anyway, n=19
  res = hypotest\_fun\_out(*samples, **kwds)
    \end{Verbatim}
 
            
\prompt{Out}{outcolor}{16}{}
    
    \begin{center}
\begin{tabular}{lclc}
\toprule
\textbf{Dep. Variable:}    &  GDP (billions)  & \textbf{  R-squared:         } &     0.949   \\
\textbf{Model:}            &       OLS        & \textbf{  Adj. R-squared:    } &     0.945   \\
\textbf{Method:}           &  Least Squares   & \textbf{  F-statistic:       } &     313.1   \\
\textbf{Date:}             & Sun, 10 Nov 2024 & \textbf{  Prob (F-statistic):} &  2.19e-12   \\
\textbf{Time:}             &     15:19:39     & \textbf{  Log-Likelihood:    } &   -135.14   \\
\textbf{No. Observations:} &          19      & \textbf{  AIC:               } &     274.3   \\
\textbf{Df Residuals:}     &          17      & \textbf{  BIC:               } &     276.2   \\
\textbf{Df Model:}         &           1      & \textbf{                     } &             \\
\textbf{Covariance Type:}  &    nonrobust     & \textbf{                     } &             \\
\bottomrule
\end{tabular}
\begin{tabular}{lcccccc}
                           & \textbf{coef} & \textbf{std err} & \textbf{t} & \textbf{P$> |$t$|$} & \textbf{[0.025} & \textbf{0.975]}  \\
\midrule
\textbf{const}             &     -55.0648  &     1335.336     &    -0.041  &         0.968        &    -2872.378    &     2762.249     \\
\textbf{Lagged Employment} &     144.5184  &        8.167     &    17.695  &         0.000        &      127.287    &      161.750     \\
\bottomrule
\end{tabular}
\begin{tabular}{lclc}
\textbf{Omnibus:}       & 35.615 & \textbf{  Durbin-Watson:     } &    0.803  \\
\textbf{Prob(Omnibus):} &  0.000 & \textbf{  Jarque-Bera (JB):  } &   87.870  \\
\textbf{Skew:}          & -2.871 & \textbf{  Prob(JB):          } & 8.30e-20  \\
\textbf{Kurtosis:}      & 11.833 & \textbf{  Cond. No.          } & 3.03e+03  \\
\bottomrule
\end{tabular}
%\caption{OLS Regression Results}
\end{center}

Notes: \newline
 [1] Standard Errors assume that the covariance matrix of the errors is correctly specified. \newline
 [2] The condition number is large, 3.03e+03. This might indicate that there are \newline
 strong multicollinearity or other numerical problems.

    

    \hypertarget{significance-test-at-the-.05-significance-level-does-employment-in-the-previous-quarter-significantly-influence-gdp-support-your-answer-using-the-regression-output.}{%
\subsubsection{Significance Test: At the .05 significance level, does
employment in the previous quarter significantly influence GDP? Support
your answer using the regression
output.}\label{significance-test-at-the-.05-significance-level-does-employment-in-the-previous-quarter-significantly-influence-gdp-support-your-answer-using-the-regression-output.}}

\begin{itemize}
\item
  Univariate linear regression:\\
  =\textgreater{} If we look into the p-value of the ``Lagged
  Employment'', it has the value of 0 which is less than 0.05. It
  signifies that ``Lagged Employment'' has significance in determining
  the ``GDP (Billions)''.
\item
  Multivariate linear regression:\\
  =\textgreater{} If we look into the p-value of the ``Lagged
  Employment'', it has the value of 0.054 which is greater than 0.05. It
  means that ``Lagged Employment'' has less significance in determining
  the ``GDP(Billions)'' when comparing with other predictors.
\end{itemize}

    \hypertarget{variation-explained-what-percentage-of-the-variation-in-gdp-is-explained-by-employment-in-the-previous-quarter}{%
\subsubsection{Variation Explained: What percentage of the variation in
GDP is explained by employment in the previous
quarter?}\label{variation-explained-what-percentage-of-the-variation-in-gdp-is-explained-by-employment-in-the-previous-quarter}}

\begin{itemize}
\item
  Univariate Linear Regression:\\
  =\textgreater{} While looking into the R-squared value, it has value
  of 0.949. This means Employment in the previous quarter can explain
  the 94.9\% of variation in GDP.
\item
  Multivariate Linear Regression:\\
  =\textgreater{} It has R-square value of 0.998. This means 99.8 \% of
  the variation in GDP is explained by predictors. Predictors are
  ``Year'', ``Lagged Employment'', ``One hot encoded quarters''. But,
  ``Lagged Employment'' has the higher p-value among all predictors.
\end{itemize}

    \hypertarget{impact-of-employment-increase-estimate-the-change-in-quarterly-gdp-for-a-1-million-increase-in-employment.}{%
\subsubsection{Impact of Employment Increase: Estimate the change in
quarterly GDP for a 1 million increase in
employment.}\label{impact-of-employment-increase-estimate-the-change-in-quarterly-gdp-for-a-1-million-increase-in-employment.}}

\begin{itemize}
\tightlist
\item
  Univariate linear regression:\\
  =\textgreater{} We only have ``Lagged Employment'' as the predictors.
  Lagged Employment is in Millions and GDP is in billion. It has the
  coefficient of around 144.5. This means for 1 million increase in
  employment, there will be 144.5 billion increase in GDP.
\item
  Multivariate linear regression:\\
  =\textgreater{} We have ``Year'', ``Lagged Employment (Millions)'',
  and One hot encoded quarters as predictors. GDP is in billion.
  ``Lagged Employment'' has the coefficient of around -19.35. This means
  they have inverse associations. 1 Million increase in employment leads
  to 19.35 billion decrease in GDP.
\end{itemize}

    \hypertarget{smartwatch_pricing}{%
\section{2. Smartwatch\_Pricing}\label{smartwatch_pricing}}

Initially, We perform EDA and prepared multivariate linear regression
model.

    \begin{tcolorbox}[breakable, size=fbox, boxrule=1pt, pad at break*=1mm,colback=cellbackground, colframe=cellborder]
\prompt{In}{incolor}{17}{\boxspacing}
\begin{Verbatim}[commandchars=\\\{\}]
\PY{n}{smartwatch\PYZus{}price\PYZus{}df} \PY{o}{=} \PY{n}{pd}\PY{o}{.}\PY{n}{read\PYZus{}excel}\PY{p}{(}\PY{l+s+s2}{\PYZdq{}}\PY{l+s+s2}{./HW2\PYZus{}Datasets.xlsx}\PY{l+s+s2}{\PYZdq{}}\PY{p}{,} \PY{n}{sheet\PYZus{}name}\PY{o}{=} \PY{l+s+s2}{\PYZdq{}}\PY{l+s+s2}{Smartwatch\PYZus{}Pricing}\PY{l+s+s2}{\PYZdq{}}\PY{p}{)}
\PY{n}{smartwatch\PYZus{}price\PYZus{}df}\PY{o}{.}\PY{n}{head}\PY{p}{(}\PY{p}{)}
\end{Verbatim}
\end{tcolorbox}

            \begin{tcolorbox}[breakable, size=fbox, boxrule=.5pt, pad at break*=1mm, opacityfill=0]
\prompt{Out}{outcolor}{17}{\boxspacing}
\begin{Verbatim}[commandchars=\\\{\}]
   Price (USD)  Battery Life (hours)  Display Quality (1-5)  \textbackslash{}
0          250                    24                      4
1          200                    18                      3
2          300                    30                      5
3          150                    16                      3
4          400                    36                      5

   Water Resistance (meters)
0                         50
1                         30
2                        100
3                         20
4                        150
\end{Verbatim}
\end{tcolorbox}
        
    \begin{tcolorbox}[breakable, size=fbox, boxrule=1pt, pad at break*=1mm,colback=cellbackground, colframe=cellborder]
\prompt{In}{incolor}{18}{\boxspacing}
\begin{Verbatim}[commandchars=\\\{\}]
\PY{n}{smartwatch\PYZus{}price\PYZus{}df}\PY{o}{.}\PY{n}{describe}\PY{p}{(}\PY{p}{)}
\end{Verbatim}
\end{tcolorbox}

            \begin{tcolorbox}[breakable, size=fbox, boxrule=.5pt, pad at break*=1mm, opacityfill=0]
\prompt{Out}{outcolor}{18}{\boxspacing}
\begin{Verbatim}[commandchars=\\\{\}]
       Price (USD)  Battery Life (hours)  Display Quality (1-5)  \textbackslash{}
count    20.000000             20.000000              20.000000
mean    298.000000             27.600000               3.850000
std      89.639629              7.315449               0.988087
min     140.000000             15.000000               2.000000
25\%     242.500000             21.750000               3.000000
50\%     297.500000             28.500000               4.000000
75\%     367.500000             33.250000               5.000000
max     430.000000             40.000000               5.000000

       Water Resistance (meters)
count                  20.000000
mean                   78.250000
std                    46.205348
min                    10.000000
25\%                    38.750000
50\%                    70.000000
75\%                   112.500000
max                   160.000000
\end{Verbatim}
\end{tcolorbox}
        
    \begin{tcolorbox}[breakable, size=fbox, boxrule=1pt, pad at break*=1mm,colback=cellbackground, colframe=cellborder]
\prompt{In}{incolor}{19}{\boxspacing}
\begin{Verbatim}[commandchars=\\\{\}]
\PY{n}{smartwatch\PYZus{}price\PYZus{}df}\PY{o}{.}\PY{n}{info}\PY{p}{(}\PY{p}{)}
\end{Verbatim}
\end{tcolorbox}

    \begin{Verbatim}[commandchars=\\\{\}]
<class 'pandas.core.frame.DataFrame'>
RangeIndex: 20 entries, 0 to 19
Data columns (total 4 columns):
 \#   Column                     Non-Null Count  Dtype
---  ------                     --------------  -----
 0   Price (USD)                20 non-null     int64
 1   Battery Life (hours)       20 non-null     int64
 2   Display Quality (1-5)      20 non-null     int64
 3   Water Resistance (meters)  20 non-null     int64
dtypes: int64(4)
memory usage: 772.0 bytes
    \end{Verbatim}

    \begin{tcolorbox}[breakable, size=fbox, boxrule=1pt, pad at break*=1mm,colback=cellbackground, colframe=cellborder]
\prompt{In}{incolor}{20}{\boxspacing}
\begin{Verbatim}[commandchars=\\\{\}]
\PY{n}{X\PYZus{}train\PYZus{}sw} \PY{o}{=} \PY{n}{smartwatch\PYZus{}price\PYZus{}df}\PY{o}{.}\PY{n}{drop}\PY{p}{(}\PY{n}{columns}\PY{o}{=} \PY{p}{[}\PY{l+s+s2}{\PYZdq{}}\PY{l+s+s2}{Price (USD)}\PY{l+s+s2}{\PYZdq{}}\PY{p}{]}\PY{p}{)}
\PY{n}{y\PYZus{}train\PYZus{}sw} \PY{o}{=} \PY{n}{smartwatch\PYZus{}price\PYZus{}df}\PY{p}{[}\PY{p}{[}\PY{l+s+s2}{\PYZdq{}}\PY{l+s+s2}{Price (USD)}\PY{l+s+s2}{\PYZdq{}}\PY{p}{]}\PY{p}{]}

\PY{n}{X\PYZus{}train\PYZus{}sw\PYZus{}with\PYZus{}const} \PY{o}{=} \PY{n}{sm}\PY{o}{.}\PY{n}{add\PYZus{}constant}\PY{p}{(}\PY{n}{X\PYZus{}train\PYZus{}sw}\PY{p}{)}

\PY{n}{sw\PYZus{}ols\PYZus{}model} \PY{o}{=} \PY{n}{sm}\PY{o}{.}\PY{n}{OLS}\PY{p}{(}\PY{n}{y\PYZus{}train\PYZus{}sw}\PY{p}{,} \PY{n}{X\PYZus{}train\PYZus{}sw\PYZus{}with\PYZus{}const}\PY{p}{)}\PY{o}{.}\PY{n}{fit}\PY{p}{(}\PY{p}{)}
\end{Verbatim}
\end{tcolorbox}

    \begin{tcolorbox}[breakable, size=fbox, boxrule=1pt, pad at break*=1mm,colback=cellbackground, colframe=cellborder]
\prompt{In}{incolor}{21}{\boxspacing}
\begin{Verbatim}[commandchars=\\\{\}]
\PY{n}{sw\PYZus{}ols\PYZus{}model}\PY{o}{.}\PY{n}{summary}\PY{p}{(}\PY{p}{)}
\end{Verbatim}
\end{tcolorbox}
 
            
\prompt{Out}{outcolor}{21}{}
    
    \begin{center}
\begin{tabular}{lclc}
\toprule
\textbf{Dep. Variable:}            &   Price (USD)    & \textbf{  R-squared:         } &     0.925   \\
\textbf{Model:}                    &       OLS        & \textbf{  Adj. R-squared:    } &     0.911   \\
\textbf{Method:}                   &  Least Squares   & \textbf{  F-statistic:       } &     65.52   \\
\textbf{Date:}                     & Sun, 10 Nov 2024 & \textbf{  Prob (F-statistic):} &  3.32e-09   \\
\textbf{Time:}                     &     15:19:39     & \textbf{  Log-Likelihood:    } &   -91.915   \\
\textbf{No. Observations:}         &          20      & \textbf{  AIC:               } &     191.8   \\
\textbf{Df Residuals:}             &          16      & \textbf{  BIC:               } &     195.8   \\
\textbf{Df Model:}                 &           3      & \textbf{                     } &             \\
\textbf{Covariance Type:}          &    nonrobust     & \textbf{                     } &             \\
\bottomrule
\end{tabular}
\begin{tabular}{lcccccc}
                                   & \textbf{coef} & \textbf{std err} & \textbf{t} & \textbf{P$> |$t$|$} & \textbf{[0.025} & \textbf{0.975]}  \\
\midrule
\textbf{const}                     &     -80.8143  &       56.929     &    -1.420  &         0.175        &     -201.499    &       39.870     \\
\textbf{Battery Life (hours)}      &      12.9974  &        3.057     &     4.252  &         0.001        &        6.518    &       19.477     \\
\textbf{Display Quality (1-5)}     &      13.2338  &        8.426     &     1.571  &         0.136        &       -4.629    &       31.096     \\
\textbf{Water Resistance (meters)} &      -0.3944  &        0.512     &    -0.771  &         0.452        &       -1.479    &        0.690     \\
\bottomrule
\end{tabular}
\begin{tabular}{lclc}
\textbf{Omnibus:}       &  1.532 & \textbf{  Durbin-Watson:     } &    2.622  \\
\textbf{Prob(Omnibus):} &  0.465 & \textbf{  Jarque-Bera (JB):  } &    1.186  \\
\textbf{Skew:}          &  0.388 & \textbf{  Prob(JB):          } &    0.553  \\
\textbf{Kurtosis:}      &  2.095 & \textbf{  Cond. No.          } &     901.  \\
\bottomrule
\end{tabular}
%\caption{OLS Regression Results}
\end{center}

Notes: \newline
 [1] Standard Errors assume that the covariance matrix of the errors is correctly specified.

    

    \hypertarget{model-significance-is-the-overall-model-significant-at-the-0.05-level-explain-your-reasoning.}{%
\subsubsection{Model Significance: Is the overall model significant at
the 0.05 level? Explain your
reasoning.}\label{model-significance-is-the-overall-model-significant-at-the-0.05-level-explain-your-reasoning.}}

=\textgreater{} F-statistic signifies the overall model significant. For
out multivariate linear regression model, It has F-statistics value of
\(3.32 * 10^{-9}\) which is much smaller than 0.05. This means overall
model has significance at the 0.05 level.

    \hypertarget{predicted-price-use-the-model-to-predict-the-price-of-the-new-smartwatch}{%
\subsubsection{Predicted Price: Use the model to predict the price of
the new
smartwatch}\label{predicted-price-use-the-model-to-predict-the-price-of-the-new-smartwatch}}

Tech Innovate expects their new product to have 20 hours of battery
life, a display quality of 4, and 30 meters of water resistance.\\
We predict the price of the new product by using the linear regression
model we trained.

    \begin{tcolorbox}[breakable, size=fbox, boxrule=1pt, pad at break*=1mm,colback=cellbackground, colframe=cellborder]
\prompt{In}{incolor}{22}{\boxspacing}
\begin{Verbatim}[commandchars=\\\{\}]
\PY{n}{new\PYZus{}smartwatch\PYZus{}with\PYZus{}const} \PY{o}{=} \PY{n}{pd}\PY{o}{.}\PY{n}{DataFrame}\PY{p}{(}\PY{p}{\PYZob{}}
    \PY{l+s+s2}{\PYZdq{}}\PY{l+s+s2}{const}\PY{l+s+s2}{\PYZdq{}}\PY{p}{:} \PY{p}{[}\PY{l+m+mi}{1}\PY{p}{]}\PY{p}{,}
    \PY{l+s+s2}{\PYZdq{}}\PY{l+s+s2}{Battery Life (hours)}\PY{l+s+s2}{\PYZdq{}}\PY{p}{:} \PY{p}{[}\PY{l+m+mi}{20}\PY{p}{]}\PY{p}{,}
    \PY{l+s+s2}{\PYZdq{}}\PY{l+s+s2}{Display Quality (1\PYZhy{}5)}\PY{l+s+s2}{\PYZdq{}}\PY{p}{:} \PY{p}{[}\PY{l+m+mi}{4}\PY{p}{]}\PY{p}{,}
    \PY{l+s+s2}{\PYZdq{}}\PY{l+s+s2}{Water Resistance (meters)}\PY{l+s+s2}{\PYZdq{}}\PY{p}{:} \PY{p}{[}\PY{l+m+mi}{30}\PY{p}{]}
\PY{p}{\PYZcb{}}\PY{p}{)}

\PY{n}{new\PYZus{}sw\PYZus{}predicted\PYZus{}price} \PY{o}{=} \PY{n}{sw\PYZus{}ols\PYZus{}model}\PY{o}{.}\PY{n}{predict}\PY{p}{(}\PY{n}{new\PYZus{}smartwatch\PYZus{}with\PYZus{}const}\PY{p}{)}

\PY{n+nb}{print}\PY{p}{(}\PY{l+s+sa}{f}\PY{l+s+s2}{\PYZdq{}}\PY{l+s+s2}{Predicted price for the new smartwatch: \PYZdl{}}\PY{l+s+si}{\PYZob{}}\PY{n}{new\PYZus{}sw\PYZus{}predicted\PYZus{}price}\PY{o}{.}\PY{n}{iloc}\PY{p}{[}\PY{l+m+mi}{0}\PY{p}{]}\PY{l+s+si}{:}\PY{l+s+s2}{.2f}\PY{l+s+si}{\PYZcb{}}\PY{l+s+s2}{\PYZdq{}}\PY{p}{)}
\end{Verbatim}
\end{tcolorbox}

    \begin{Verbatim}[commandchars=\\\{\}]
Predicted price for the new smartwatch: \$220.24
    \end{Verbatim}

    \hypertarget{significant-drivers-based-on-the-model-which-factors-are-significant-drivers-of-price-are-there-any-variables-that-seem-counterintuitive}{%
\subsubsection{Significant Drivers: Based on the model, which factors
are significant drivers of price? Are there any variables that seem
counterintuitive?}\label{significant-drivers-based-on-the-model-which-factors-are-significant-drivers-of-price-are-there-any-variables-that-seem-counterintuitive}}

Variable

coef

std err

t

P\textgreater\textbar t\textbar{}

const

-80.8143

56.929

-1.420

0.175

Battery Life (hours)

12.9974

3.057

4.252

0.001

Display Quality (1-5)

13.2338

8.426

1.571

0.136

Water Resistance (meters)

-0.3944

0.512

-0.771

0.452

As we can see in the table above, ``Battery Life (hours)'' has least
p-value which means it is the most significant factor while determining
variance in price. After that, ``Display Quality'' has the moderate
significance and ``Water Resistance (meters)'' has least significance.\\
\strut \\
While looking into the coefficients of 3 predictors, - ``Battery Life
(hours)'' and ``Display Quality (1-5)'' has positive coefficient values.
This means whenever battery life or display quality increases, it
increases the price of the watch. - But, ``Water Resistance (meters)''
has negative coefficient. It is the counterintuitive variable. whenever,
water resistance increases, it tends to decrease the price of watch.

    \hypertarget{correlation-analysis-check-for-multicollinearity-by-calculating-the-correlations-between-the-drivers.-which-pairs-of-variables-show-a-strong-correlation}{%
\subsubsection{Correlation Analysis: Check for multicollinearity by
calculating the correlations between the drivers. Which pairs of
variables show a strong
correlation?}\label{correlation-analysis-check-for-multicollinearity-by-calculating-the-correlations-between-the-drivers.-which-pairs-of-variables-show-a-strong-correlation}}

    \begin{tcolorbox}[breakable, size=fbox, boxrule=1pt, pad at break*=1mm,colback=cellbackground, colframe=cellborder]
\prompt{In}{incolor}{23}{\boxspacing}
\begin{Verbatim}[commandchars=\\\{\}]
\PY{c+c1}{\PYZsh{} Price field dropped as we are trying to check multicollinearity among predictors}
\PY{n}{sw\PYZus{}correlation\PYZus{}matrix} \PY{o}{=} \PY{n}{smartwatch\PYZus{}price\PYZus{}df}\PY{o}{.}\PY{n}{drop}\PY{p}{(}\PY{n}{columns}\PY{o}{=} \PY{p}{[}\PY{l+s+s2}{\PYZdq{}}\PY{l+s+s2}{Price (USD)}\PY{l+s+s2}{\PYZdq{}}\PY{p}{]}\PY{p}{)}\PY{o}{.}\PY{n}{corr}\PY{p}{(}\PY{p}{)}
\PY{n}{sw\PYZus{}correlation\PYZus{}matrix}
\end{Verbatim}
\end{tcolorbox}

            \begin{tcolorbox}[breakable, size=fbox, boxrule=.5pt, pad at break*=1mm, opacityfill=0]
\prompt{Out}{outcolor}{23}{\boxspacing}
\begin{Verbatim}[commandchars=\\\{\}]
                           Battery Life (hours)  Display Quality (1-5)  \textbackslash{}
Battery Life (hours)                   1.000000               0.617456
Display Quality (1-5)                  0.617456               1.000000
Water Resistance (meters)              0.960879               0.668344

                           Water Resistance (meters)
Battery Life (hours)                        0.960879
Display Quality (1-5)                       0.668344
Water Resistance (meters)                   1.000000
\end{Verbatim}
\end{tcolorbox}
        
    \begin{tcolorbox}[breakable, size=fbox, boxrule=1pt, pad at break*=1mm,colback=cellbackground, colframe=cellborder]
\prompt{In}{incolor}{24}{\boxspacing}
\begin{Verbatim}[commandchars=\\\{\}]
\PY{n}{sns}\PY{o}{.}\PY{n}{heatmap}\PY{p}{(}\PY{n}{data}\PY{o}{=} \PY{n}{sw\PYZus{}correlation\PYZus{}matrix}\PY{p}{,} \PY{n}{annot}\PY{o}{=} \PY{k+kc}{True}\PY{p}{,} \PY{n}{cmap}\PY{o}{=} \PY{l+s+s2}{\PYZdq{}}\PY{l+s+s2}{YlGnBu}\PY{l+s+s2}{\PYZdq{}}\PY{p}{)}
\PY{n}{plt}\PY{o}{.}\PY{n}{xticks}\PY{p}{(}\PY{n}{rotation}\PY{o}{=}\PY{l+m+mi}{45}\PY{p}{)}
\PY{n}{plt}\PY{o}{.}\PY{n}{yticks}\PY{p}{(}\PY{n}{rotation}\PY{o}{=}\PY{l+m+mi}{45}\PY{p}{)}
\PY{n}{plt}\PY{o}{.}\PY{n}{title}\PY{p}{(}\PY{l+s+s1}{\PYZsq{}}\PY{l+s+s1}{Correlation Matrix of Predictors and heatmap}\PY{l+s+s1}{\PYZsq{}}\PY{p}{)}
\PY{n}{plt}\PY{o}{.}\PY{n}{show}\PY{p}{(}\PY{p}{)}
\end{Verbatim}
\end{tcolorbox}

    \begin{center}
    \adjustimage{max size={0.9\linewidth}{0.9\paperheight}}{E923Q669_assignment2_files/E923Q669_assignment2_36_0.png}
    \end{center}
    { \hspace*{\fill} \\}
    
    According to the above correlation matrices, it is found that - battery
life and water resistance has the maximum correlation coefficient of
0.96. It means there is strong association and collinearity between
``Battery Life (hours)'' and ``Water Resistance (meter)''.

    \hypertarget{revised-model-if-necessary-remove-any-highly-correlated-drivers-and-rerun-the-regression.-has-the-model-improved}{%
\subsubsection{Revised Model: If necessary, remove any highly correlated
drivers and rerun the regression. Has the model
improved?}\label{revised-model-if-necessary-remove-any-highly-correlated-drivers-and-rerun-the-regression.-has-the-model-improved}}

=\textgreater{} As we can see, there is high correlation between
``Battery Life (hours)'' and ``Water Resistance (meter)'', we will be
removing ``Water Resistance (meter)'' from predictors and preparing new
linear regression model.

    \begin{tcolorbox}[breakable, size=fbox, boxrule=1pt, pad at break*=1mm,colback=cellbackground, colframe=cellborder]
\prompt{In}{incolor}{25}{\boxspacing}
\begin{Verbatim}[commandchars=\\\{\}]
\PY{n}{new\PYZus{}X\PYZus{}train\PYZus{}sw} \PY{o}{=} \PY{n}{smartwatch\PYZus{}price\PYZus{}df}\PY{o}{.}\PY{n}{drop}\PY{p}{(}\PY{n}{columns}\PY{o}{=} \PY{p}{[}\PY{l+s+s2}{\PYZdq{}}\PY{l+s+s2}{Price (USD)}\PY{l+s+s2}{\PYZdq{}}\PY{p}{,} \PY{l+s+s2}{\PYZdq{}}\PY{l+s+s2}{Water Resistance (meters)}\PY{l+s+s2}{\PYZdq{}}\PY{p}{]}\PY{p}{)}

\PY{n}{new\PYZus{}X\PYZus{}train\PYZus{}sw\PYZus{}with\PYZus{}const} \PY{o}{=} \PY{n}{sm}\PY{o}{.}\PY{n}{add\PYZus{}constant}\PY{p}{(}\PY{n}{new\PYZus{}X\PYZus{}train\PYZus{}sw}\PY{p}{)}

\PY{n}{new\PYZus{}sw\PYZus{}ols\PYZus{}model} \PY{o}{=} \PY{n}{sm}\PY{o}{.}\PY{n}{OLS}\PY{p}{(}\PY{n}{y\PYZus{}train\PYZus{}sw}\PY{p}{,} \PY{n}{new\PYZus{}X\PYZus{}train\PYZus{}sw\PYZus{}with\PYZus{}const}\PY{p}{)}\PY{o}{.}\PY{n}{fit}\PY{p}{(}\PY{p}{)}
\end{Verbatim}
\end{tcolorbox}

    \begin{tcolorbox}[breakable, size=fbox, boxrule=1pt, pad at break*=1mm,colback=cellbackground, colframe=cellborder]
\prompt{In}{incolor}{26}{\boxspacing}
\begin{Verbatim}[commandchars=\\\{\}]
\PY{n}{new\PYZus{}sw\PYZus{}ols\PYZus{}model}\PY{o}{.}\PY{n}{summary}\PY{p}{(}\PY{p}{)}
\end{Verbatim}
\end{tcolorbox}
 
            
\prompt{Out}{outcolor}{26}{}
    
    \begin{center}
\begin{tabular}{lclc}
\toprule
\textbf{Dep. Variable:}        &   Price (USD)    & \textbf{  R-squared:         } &     0.922   \\
\textbf{Model:}                &       OLS        & \textbf{  Adj. R-squared:    } &     0.913   \\
\textbf{Method:}               &  Least Squares   & \textbf{  F-statistic:       } &     100.4   \\
\textbf{Date:}                 & Sun, 10 Nov 2024 & \textbf{  Prob (F-statistic):} &  3.85e-10   \\
\textbf{Time:}                 &     15:19:40     & \textbf{  Log-Likelihood:    } &   -92.280   \\
\textbf{No. Observations:}     &          20      & \textbf{  AIC:               } &     190.6   \\
\textbf{Df Residuals:}         &          17      & \textbf{  BIC:               } &     193.5   \\
\textbf{Df Model:}             &           2      & \textbf{                     } &             \\
\textbf{Covariance Type:}      &    nonrobust     & \textbf{                     } &             \\
\bottomrule
\end{tabular}
\begin{tabular}{lcccccc}
                               & \textbf{coef} & \textbf{std err} & \textbf{t} & \textbf{P$> |$t$|$} & \textbf{[0.025} & \textbf{0.975]}  \\
\midrule
\textbf{const}                 &     -42.1472  &       26.583     &    -1.585  &         0.131        &      -98.233    &       13.939     \\
\textbf{Battery Life (hours)}  &      10.7902  &        1.056     &    10.222  &         0.000        &        8.563    &       13.017     \\
\textbf{Display Quality (1-5)} &      10.9969  &        7.815     &     1.407  &         0.177        &       -5.492    &       27.486     \\
\bottomrule
\end{tabular}
\begin{tabular}{lclc}
\textbf{Omnibus:}       &  1.555 & \textbf{  Durbin-Watson:     } &    2.494  \\
\textbf{Prob(Omnibus):} &  0.460 & \textbf{  Jarque-Bera (JB):  } &    1.287  \\
\textbf{Skew:}          &  0.464 & \textbf{  Prob(JB):          } &    0.525  \\
\textbf{Kurtosis:}      &  2.172 & \textbf{  Cond. No.          } &     130.  \\
\bottomrule
\end{tabular}
%\caption{OLS Regression Results}
\end{center}

Notes: \newline
 [1] Standard Errors assume that the covariance matrix of the errors is correctly specified.

    

    Yes, Removing the highly correlated predictor (In this case Water
Resistance) has improved the model.

Newly trained linear regression model has the performance similar to
original model. - New model has R-squared value of 0.922 which mean
Battery life and Display quality can explain the 92.2\% variance in
price. But in old model, there is 0.925 R-squared value which mean
Battery life, Display quality and Water Resistance (meters) 92.5\%
variance in price. - If we look into the adjusted R-squared value of
both old model and new model, new model has higher adjusted R-squared
value of 0.913 in comparision with old model which has adjusted
R-squared value of 0.911. This mean, removing water resistance variable
as predictor actually improves the predictive ability of the model.


    % Add a bibliography block to the postdoc
    
    
    
\end{document}
